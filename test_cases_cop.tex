\documentclass[sigconf]{acmart}
\usepackage{balance}
\usepackage{graphicx}
\usepackage{url}
\usepackage{amsmath}
\usepackage{mathtools}
\usepackage{tabularx}
\usepackage{caption}
\usepackage{subcaption}
\usepackage{multirow}
\usepackage{graphics}


\begin{document}

\section{Introduction:}
\par I am a student of Computer Science and Engineering department at \textbf{IIT Delhi}. I am pursuing my \underline{MTech} course.My stay at \textit{IIT Delhi} has been pretty enthralling and engrossing till date.

\subsection{The courses I have taken in the first year are as follows:}
\begin{enumerate}
	\item Advanced Data Structures
	\item Functional Programming
	\item Software Systems Lab
	\item Artificial Intelligence
	\item Special Topics in AI
	\item Machine Learning
	\item Topics in Multimedia
	\item Minor project
\end{enumerate}

\subsection{Our Building} \label{introduction}
This the view of the main building of IIT Delhi.

\begin{figure}[h]
\includegraphics[width=250,height=250]{iitd.jpg}
\caption{This image will be referenced below}
\label{fig:IITD Building}
\end{figure}

You can reference images, for instance, the image \ref{fig:IITD Building} shows
the image of \textit{IIT Delhi Building}

\section{Testing Tabular environment:}

\begin{tabular}{ |c|c|c| }
 \hline
 cell1 & cell2 & cell3 \\
 \hline
 cell4 & cell5 & cell6 \\
 \hline
 cell7 & cell8 & \begin{figure}[h]
	\includegraphics[width=250,height=250]{iitd.jpg}
	\end{figure}
	 \\
 \hline
\end{tabular}

\section{Testing Math Symbols:}
\subsection{Frac} $x=(a+b)/2$
\subsection{sqrt} $x=\frac{b}{-4ac}$
\subsection{sqrt} $y=\sqrt{(a+b)}$

\end{document}
